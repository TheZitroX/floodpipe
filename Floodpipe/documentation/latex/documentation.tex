\documentclass[12pt, letterpaper]{article}

% Usages
\usepackage[utf8]{inputenc}
\usepackage[ngerman]{babel}
% ?muss das sein?
\usepackage[T1]{fontenc}
% !austesten
\usepackage{hyphenat}
\usepackage{lmodern}

\begin{document}

% sections
% -1\part{part}
% 0	\chapter{chapter}
% 1	\section{section}
% 2	\subsection{subsection}
% 3	\subsubsection{subsubsection}
% 4	\paragraph{paragraph}
% 5	\subparagraph{subparagraph}

    % Cover of the document
    % todo better cover
    \title{Delphi Programmierpraktikum Floodpipe}
    \author{Lienau, John - Informationstechnischer Assistent (104923)}
    \date{\today}
    \maketitle
    \newpage

    \tableofcontents
    \newpage

    % Zusammenfassung des Textes
    % \begin{abstract}
    %     Dies ist eine kurze Zusammenfassung der Inhalte des in deutscher Sprache verfassten Dokuments.
    %     Einige Umlaute mit ä,ö,ü - welche immer probleme machen!
    % \end{abstract}
    % \newpage

    \section{Allgemeine Problemstellung}
        Im Benutzerhandbuch sollte kurz erläutert werden, worum es sich bei Eurem Programm eigendlich handelt
        - bitte \textbf{kein Copy\&Paste} der Aufgabenstellung!

        \subsection{Anforderungen}
            In diesem Kapitel beschreibt Ihr so detailliert wie möglich, welche Anforderungen an das Programm gestellt werden
            (welche Funktionalitäten das Programm bereitstellen soll) und wieso.
            Zusätzlich beinhaltet dieses Kapitel auch, für welche Anwender-Zielgruppe das Programm gedacht ist,
            warum jemand das Programm nutzen sollte, welche Realwelt-Probleme damit gelöst werden.

            Das Kapitel wird unabhängig von der tatsächlichen Implementierung und unabhängig von
            der programmtechnischen Umsetzung in Pascal formuliert. 
            Schreibt das Kapitel so, als würdet Ihr einer anderen Person die zu lösende Aufgabe beschreiben. 
            \textbf{Umfang: etwa eine vollständige DINA4-Seite.}
    \newpage

    \section{Benutzerhandbuch}
    Das Benutzerhandbuch besteht mindestens
    (ohne Ausnahme) aus folgenden Kapiteln
    (Kapitelüberschriften müssen \textbf{genau so} beibehalten werden): 
            
        \subsection{Themen}
            We have now added a title, author and date to our first \LaTeX{} document!

            Hello World!
            My first Latex article

            Lorem ipsum dolor sit amet, consectetuer adipiscing elit.  
            Etiam lobortis facilisissem.  Nullam nec mi et neque pharetra 
            sollicitudin.  Praesent imperdiet mi necante...

            \subsubsection{Thema 1}
            Praesent imperdietmi nec ante. Donec ullamcorper, felis non sodales...
    
        \subsection{Ablaufbedingungen}
            Eine saubere und vollständige tabellarische Auflistung der Hardware und Softwarekomponenten (inkl. Version),
            die für den Ablauf der Exe-Version Eures Programms mindestens notwendig sind.
            In der Regel ist hier lediglich ein 32-Bit-Windows aufzuführen.

        \subsection{Programminstallation}
            Welche Dateien muss der Benutzer zum Programmablauf genau wohin kopieren (Quelle und Ziel angeben), welche Zusatzkomponenten müssen evtl. installiert werden oder welche weiteren Vorbereitungen müssen noch getroffen werden, bevor das Programm ausgeführt werden kann. Bezieht ein Programm beispielsweise Bilddateien aus einem relativ gelegenen Unterverzeichnis und legt Einstellungswerte in einer Datei im Verzeichnis des Programms an, so muss das Bildverzeichnis und seine relative Lage angegeben werden und auf die notwendigen Schreibrechte im Verzeichnis des ausgeführten Programms hingewiesen werden.
        
        \subsection{Programmstart}
            Was genau muss man tun, damit das Programm startet?

        \subsection{Bedienungsanleitung}
            Zuerst mal werden hier die Hintergrundinformationen, die für das Verständnis des Programms notwendig sind, aufgeführt. Dann wird jede Funktionalität des Programms und jedes mögliche Ablaufstadium sorgfältig und so detailliert wie möglich erklärt. Dabei sollte von Screenshots ausführlich Gebrauch gemacht werden (diese sollen sinnvoll in den Fließtext eingebunden werden). Auch können z.B. Spielregeln aus der Aufgabenstellung kopiert in den Text einfließen. Eine weitere Untergliederung, inkl. Überschriften und Aufnahme in das Inhaltsverzeichnis, ist zwingend erforderlich.

        \subsection{Fehlermeldungen}
            Sämtliche Fehlermeldungen, die im Programm auftreten können, werden in einer Tabelle, bestehend aus den Spalten 'Fehlermeldung', 'Fehlerursache' und 'Behebungsmaßnahme' aufgelistet.

        \subsection{Wiederanlaufbedingungen}
            Was ist zu tun, wenn das Programm zur Laufzeit (z.B. durch Stromausfall) abgebrochen wird? Sind ungespeicherte Änderungen verloren gegangen? Welche Dateien sind betroffen?
    \newpage

    \section{Programmierhandbuch}
        Grundsätzlich soll dieser Teil dazu dienen, einen Software-Entwickler über Deine Gedankengänge bei der Planung und Umsetzung des Projektes zu informieren, damit er sich nach möglichst kurzer Einarbeitungszeit der Fortführung, Pflege etc. widmen kann. Aussagekräftige und gut strukturierte Darstellungen sind hier gefordert. Stell Dir selbst einmal die Frage, was Du alles brauchst, um Dich in Deinem Projekt nach einigen Jahren wieder zurechtzufinden!

        Das Programmierhandbuch ist der wichtigste Teil der Dokumentation und besteht immer und ohne Ausnahme mindestens aus folgenden Kapiteln (Kapitelüberschriften sollten genauso beibehalten werden):

        \subsection{Entwicklungskonfiguration}
            Eine saubere und vollständige tabellarische Auflistung der Softwarekomponenten mit Hilfe derer Ihr das Programm entwickelt habt. Relevant sind hier Betriebssystem und Compiler mit Angabe der jeweiligen Version. Wurde das Programm auf mehreren unterschiedlichen Rechnern entwickelt, sind die unterschiedlichen Konfigurationen anzugeben.
        
        \subsection{Problemanalyse und Realisation}
            Folgendermaßen können die erwarteten Informationen übersichtlich strukturiert werden (grundsätzlich gelten diese Ausführungen auch für andere / ähnliche technische Dokumentationen, in denen dem Leser Hintergrundwissen zur Lösung einer Aufgabenstellung vermittelt werden soll):
            \begin{verse}
                % \begin{itemize}
                    \item Sinnvollerweise startet man mit einer Auflistung der zu erfüllenden Aufgaben, wobei auf einen ausreichenden Detaillierungsgrad geachtet werden sollte. Die einzelnen Punkte ergeben sich aus der Aufgabenstellung (und werden ggf. in Diskussion mit dem Auftraggeber weiter verfeinert).

                    \item Die Aufstellung wird nach Sachgebieten, Modulen, übergeordneten Teilaufgaben o.ä. gruppiert und fasst ggf. Detailpunkte zusammen. Hiermit erhält man eine Art „Checkliste der zu erfüllenden Aufgaben“, die während der Projektphase durchaus noch modifiziert werden kann.

                    \item Damit verfügt man über ein Dokument (evtl. Teil eines Lasten- /Pflichtenheftes), das die wesentlichen Problemstellungen darstellt -> Problemanalyse.

                    \item In der nun folgenden Realisationsanalyse werden für jeden einzelnen Teilaspekt ein oder mehrere Lösungswege aufgezeigt und diskutiert (Pro und Contra). Das passiert häufig ohne Festlegung auf eine bestimmte Programmiersprache (hier im Programmierpraktikum gilt natürlich Delphi/Object Pascal!). In diesem Kapitel sollten also z.B. Fragen diskutiert werden wie 'ist eine iterative oder eine rekursive Lösung hier besser?', 'ist Komponente x oder y aus Delphi besser geeignet?' oder auch 'kann man diesen Punkt mit einem Array oder eine Liste besser umsetzen?'. Beschreibt hier immer objektive Vor- und Nachteile der verschiedenen Ansätze (und eben nicht 'das war für mich einfacher umzusetzen').

                    \item Aus dieser Diskussion ergibt sich (hoffentlich) die beste aller Lösungen für diesen Problempunkt und man kann den nächsten anpacken… Sind nun alle einzelnen Punkte bearbeitet und steht damit ein Lösungsweg für jeden Teilaspekt fest, so ist das Kapitel „Realisationsanalyse“ (in der ersten Version) komplettiert.

                    \item Im dritten Unterkapitel steht für jede soeben gefundene (lt. Diskussion beste) Teillösung die Darstellung der Implementierung im Vordergrund. Charts, Diagramme, Verweise auf andere Kapitel (Programmorganisationsplan, Datenstrukturen etc.) helfen dem Softwareentwickler, der sich in Dein Projekt einarbeiten muss, Deine Gedankengänge nachzuvollziehen. Und schon ist die Realisationsbeschreibung fertig!

                    \item Die Präsentationsform des Gesamtkapitels obliegt natürlich Dir, sollte aber möglichst übersichtlich und wohlstrukturiert sein. Also beispielsweise: Zu lösendes Teilproblem – pro/contra zu den möglichen Lösungsansätzen – Beschreibung der Realisation. Und dann das gleiche noch einmal für die nächsten Punkte. Wählt man diese Darstellungsform, so entfällt für den Leser das leidige Springen zwischen den drei Hauptkapiteln. Zusätzlich kann dann auch die Überschriftenfolge „Problemanalyse“, „Realisationsanalyse“ und „Realisationsbeschreibung“ zu jedem Detailpunkt entfallen.

                    \item Unabhängig von der gewählten Präsentationsform sollte jeder Teilaspekt schnell zu finden sein: gute Struktur, keine übermäßig langen Texte, Konzentration auf die wesentlichen Fakten, ggf. Aufnahme in das Inhaltsverzeichnis, sinnvolle und eindeutige Überschriften. Eine erste Version dieses Kapitels ist tunlichst vor der eigentlichen Programmierung zu erstellen und als Planungshilfe für das Projekt zu benutzen. Natürlich können nach der Planung beim Programmieren Änderungen und Erweiterungen der ursprünglichen Problemanalyse und Realisation entstehen, die dann in die Endversion des Kapitels mit übernommen werden.
                % \end{itemize}
            \end{verse}
            Um die Überlegungen auch nachvollziehbar zu machen ist eine gute Doku-Struktur mit passenden Überschriften unabdingbar (niemand liest 3-4 Seiten Fließtext).
        
        \subsection{Beschreibung grundlegender Datenstrukturen}
            Hier werden alle Datenstrukturen, die Ihr im Programm tatsächlich benutzt, aufgelistet, jeweils mit einer abstrakten Beschreibung der Struktur (bei dynamischen Datenstrukturen wird zwecks besserer Anschaulichkeit zusätzlich eine Grafik verlangt), dem Zweck, dem diese Struktur in dem Programm dient, sowie Ihr pascal-spezifischer Aufbau (also die explizite Typdefinition aus dem Programm). 
            
            Außerdem wird der Aufbau aller im Programm verwendeten Dateien (sowohl typisierte Dateien als auch Textdateien) aufgeführt. Im Gegensatz zum Kapitel Problemanalyse und Realisation, in dem die Auswahl und deren Begründung einer oder mehrerer Datenstrukturen stattfindet, soll hier der tatsächliche Einsatz und die genaue Zusammensetzung der Datenstrukturen erläutert werden.
        
        \subsection{Programmorganisationsplan}
            Stellt grafisch dar, wie sich die Units, die ihr verwendet, untereinander aufrufen. Es ist sinnvoll, in der Grafik die Units zu gruppieren (z.B. nach Formular, Logik, Daten, Typen, etc.). Dies kann zum Beispiel durch eine farbliche Hervorhebung geschehen. Zusätzlich zur Grafik sollte ein Fließtext oder eine tabellarische Darstellung die Grafik und ihre Aufteilung beschreiben.
            
        \subsection{Programmtests}
            Während und nach der Programmerstellung muss das Programm umfangreich getestet werden! Dieser Vorgang soll nicht willkürlich geschehen. Dieser Testvorgang soll systematisch durchgeführt und dokumentiert werden.
            \begin{verse}
                Es müssen Testfälle für alle Funktionalitäten und alle Eventualitäten erdacht werden. Die Tests verlaufen also systematisch nach einem Vorgehensplan (für jedes Eingabefeld z.B. werden also alle möglichen Konstellationen inkl. Falscheingaben getestet).

                Da das Programm von Euch selbst erstellt wurde, ist es Euch möglich, Testfälle abzuleiten, bei denen besondere Probleme erwartet werden. Es soll genügen, wenn nur diese speziellen Fälle dokumentiert werden (Programm kann blockieren, mögliche Dateninkonsistenzen, Verletzung von weiteren Bedingungen durch fehlerhafte Eingabe - z.B. 'Div by zero'). Dazu ist es aber notwendig, diese Fälle zu erkennen (am besten gleich beim Programmieren). 
                Diese Testfälle stellt Ihr in diesem Kapitel in einer Tabelle, bestehend aus den Spalten 'Testfall', 'erwartetes Ergebnis' und 'erzieltes Ergebnis' dar. In der Spalte 'Testfall' steht jeweils eine detaillierte Beschreibung des Testfalls (so gut beschreiben, dass man ihn leicht nachstellen kann). In der Spalte 'erwartetes Ergebnis' steht jeweils die Reaktion des Programms auf den Testfall, die Ihr erwartet habt (bei mathematischen Berechnungen also das Ergebnis), bevor Ihr den Testfall getestet habt. In der Spalte 'erzieltes Ergebnis' steht das tatsächliche Ergebnis. Der Inhalt der beiden letzten Spalten wird, wenn alles gut geht, der gleiche sein. (Bei mathematischen Rechnungen können sich aufgrund der begrenzten numerischen Genauigkeiten Abweichungen ergeben, die nicht auf einen inkorrekten Algorithmus oder fehlerhafte Algorithmusumsetzung zurückzuführen sind. Diese sind dann ggf. kurz zu begründen.) 
                
                Werden zum Testen externe Dateien (schreibgschützte Datensätze, absichtlich veränderte Dateien, spezielle Spielstände etc.) benötigt, so sind diese in der Beschreibung des Testfalles klar zu benennen (und logischerweise mitzuliefern). Ist das Programm durch das Laden von Dateien in der Lage, bestimmte Zustände einzunehmen (z.B. Laden von besonderen Spielständen), so sollte mit verschiedenen Dateien die Funktionalität wichtiger Algorithmen demonstriert werden.
                Dieses Kapitel hat also zwei Aufgaben: Erstens soll es protokollieren, wie der Test organisiert und durchgeführt wurde, zweitens sollen die Testergebnisse dargestellt werden.
            \end{verse}
        
\end{document}